%%%%%%%%%%%%%%%%%%%%%%%%%%%%%%%%%%%%%%%%%%%%%%%%%%%%%%%%%%%%%%%%%%%%%%%%%%%%%%%%%%%%%%%%%
%									  TESIS DE DOCTORADO
%								CHRISTIAN FABIAN GARCIA ROMERO
%							   UNIVERSIDAD NACIONAL DE COLOMBIA
%
%%%%%%%%%%%%%%%%%%%%%%%%%%%%%%%%%%%%%%%%%%%%%%%%%%%%%%%%%%%%%%%%%%%%%%%%%%%%%%%%%%%%%%%%%


%-----------------------------------------------------------------------------------------
% INICIO DEL DOCUMENTO
\documentclass[11pt,fleqn,openany,letterpaper,pagesize]{scrbook}

%%%%%%%%%%%%%%%%%%%%%%%%%%%%%%%%%%%%%%%%%%%%%%%%%%%%%%%%%%%%%%%%%%%%%%%%%%%%%%%%%%%%%%%%%
% 							INTRODUCCIÓN DE LOS PAQUETES NECESARIOS
\usepackage[utf8]{inputenc} % Accept different input encodings
\usepackage[spanish]{babel}\spanishlcroman % Multilingual support for Plain TEX or LATEX
\usepackage[automark,headsepline]{scrpage2} % Control of page headers and footers in LATEX
\usepackage{epsfig} % Include Encapsulated PostScript in LATEX documents
\usepackage{epic} % Enhance LATEX picture mode
\usepackage{eepic} % Extensions to epic and the LATEX drawing tools
\usepackage{here} % Emulation of obsolete package for "here" floats
\usepackage{lscape} % Place selected parts of a document in landscape
\usepackage{scrhack} %
\usepackage{csquotes} % Context sensitive quotation facilities
\usepackage{rotating} % Rotation tools
\usepackage{chngcntr} % Continue footnote
\counterwithout{footnote}{chapter}
\usepackage[footnotesep=1.2\baselineskip]{geometry}
\usepackage[bottom]{footmisc}
\usepackage{listings}
\setcounter{secnumdepth}{4}
\usepackage{enumitem}
\usepackage[T1]{fontenc}
\usepackage[scaled]{uarial}
\renewcommand*\familydefault{\sfdefault}
\usepackage{sectsty}
\chapterfont{\Huge\selectfont\bfseries}
\sectionfont{\LARGE\selectfont\bfseries}
\subsectionfont{\Large\selectfont\bfseries}
\renewcommand{\thechapter}{\arabic{chapter}}
\renewcommand{\autodot}{.}
\renewcommand*{\chapterformat}{%
 \Huge\selectfont\thechapter\autodot\enskip}
\renewcommand{\chapterpagestyle}{empty}
\usepackage{morewrites}
\usepackage{lipsum}

%----------------------------------------------------------------------------------------
% 								ECUACIONES MATEMÁTICAS
\usepackage{amsmath,nccmath} % AMS mathematical facilities for LATEX
\setlength{\mathindent}{\parindent}
\usepackage{mathtools}
\usepackage{booktabs}
\usepackage[customcolors]{hf-tikz}

\usepackage[most]{tcolorbox}

%---------------------------------------------------------------------------------------
% 										FIGURAS 
\usepackage{threeparttable} %Tables with captions and notes all the same width
\usepackage{amscd} % AMS-LATEX commutative diagrams
\usepackage{graphicx} % Enhanced support for graphics
%\usepackage{subfigure} % Deprecated: Figures divided into sub figures
\usepackage{caption}
\usepackage{subcaption}
\usepackage{chngcntr}

%---------------------------------------------------------------------------------------
% 										TABLAS
\usepackage{tabularx} % Tabulars with adjustable-width columns
\usepackage{csvsimple} % Simple CSV file processing
\usepackage{longtable} % Allow tables to flow over page boundaries
\usepackage{booktabs} %
\usepackage{bigstrut} %
\usepackage{array} %
%\usepackage[table,xcdraw]{xcolor}
\usepackage{multirow}
\usepackage{color, colortbl}
\definecolor{DarkBlue}{rgb}{0.13,0.21,0.39}
\usepackage{colortbl,hhline}
\definecolor{White}{rgb}{1,1,1}
\definecolor{mygreen}{RGB}{28,172,0} % color values Red, Green, Blue
\definecolor{mylilas}{RGB}{170,55,241}

%---------------------------------------------------------------------------------------
% 									BIBLIOGRAFÍA 
\usepackage[backend=biber,style=numeric-comp,sorting=none,url=false,language=english]{biblatex}
%\usepackage[backend=bibtex8,style=numeric-comp,sorting=none,url=false,language=english]{biblatex}
\bibliography{BiB/references.bib}
\AtEveryBibitem{\clearfield{month}}
\DefineBibliographyStrings{spanish}{andothers = {et\addabbrvspace al\adddot}}
\renewcommand*{\bibfont}{\scriptsize}

%-----------------------------------------------------------------------------
% 									REFERENCIAS
\usepackage[bookmarks=true,debug,bookmarksnumbered]{hyperref}
\hypersetup{colorlinks=true,linkcolor=black,urlcolor=blue,citecolor=blue}
\newcommand\fnurl[2]{\href{#2}{#1}\footnote{\url{#2}}}
\newcommand{\MYhref}[3][blue]{\href{#2}{\color{#1}{#3}}}

%-----------------------------------------------------------------------------
% 									GLOSARIO
\usepackage{siunitx}
\usepackage[acronym,section,nonumberlist,nopostdot,nogroupskip]{glossaries} 
%%%%%%%%%%%%%%%%%%%%%%%%%%%%%%%%%%%%%%%%%%%%%%%%%%%%%%%%%%%%%%%%%%%%%%%%%%%%%%%%%%%%%%%%%
%									TESIS DE DOCTORADO
%								CHRISTIAN FABIAN GARCIA ROMERO
%							   UNIVERSIDAD NACIONAL DE COLOMBIA
%
%%%%%%%%%%%%%%%%%%%%%%%%%%%%%%%%%%%%%%%%%%%%%%%%%%%%%%%%%%%%%%%%%%%%%%%%%%%%%%%%%%%%%%%%%


%%%%%%%%%%%%%%%%%%%%%%%%%%%%%%%%%%%%%%%%%%%%%%%%%%%%%%%%%%%%%%%%%%%%%%%%%%%%%%%%%%%%%%%%%

% LISTA DE SÍMBOLOS

%%%%%%%%%%%%%%%%%%%%%%%%%%%%%%%%%%%%%%%%%%%%%%%%%%%%%%%%%%%%%%%%%%%%%%%%%%%%%%%%%%%%%%%%%

%%%%%%%%%%%%%%%%%%%%%%%%%%%%%%%%%%%%%%%%%%%%%%%%%%%%%%%%%%%%%%%%%%%%%%%%%%%%%%%%%%%%%%%%%
% CONFIGURACIONES INICIALES DE LA TABLA

%\setlength{\glsdescwidth}{0.8\hsize}
\setlength{\glsdescwidth}{15cm}

\newglossary[sl1]{symbolsLAT}{sl2}{sl3}{Símbolos con letras latinas}
\newglossary[sg1]{symbolsGRI}{sg2}{sg3}{Símbolos con letras griegas}
\newglossary[sp1]{supIndx}{sp2}{sp3}{Superíndices}
\newglossary[sb1]{subIndx}{sb2}{sb3}{Subíndices}


\glsaddkey{unit}{\glsentrytext{\glslabel}}
{\glsentryunit}{\GLsentryunit}{\glsunit}{\Glsunit}{\GLSunit}

\glsaddkey{def}{\glsentrytext{\glslabel}}
{\glsentrydef}{\GLsentrydef}{\glsdef}{\Glsdef}{\GLSdef}

\makeglossaries

\newglossarystyle{symbunitlong}
{
    \setglossarystyle{long4col}
    \renewenvironment{theglossary}
    {
        \begin{longtable}
        	{lp{0.6\glsdescwidth}>{\arraybackslash}p{2.5cm}p{2cm}}}%
        {\end{longtable}
    }

	\renewcommand*{\glossaryheader}
    {
  		\bfseries Símbolo & \bfseries Termino & \bfseries Unidad SI & \bfseries Definición\\
  		\hline \\
  		\endhead
    }

    \renewcommand*{\glossentry}[2]
    {
        \glstarget{##1}{\glossentryname{##1}} 
        & \glossentrydesc{##1}
        & \glsunit{##1}  
        & \glsdef{##1}
        \tabularnewline
    }
}
%%%%%%%%%%%%%%%%%%%%%%%%%%%%%%%%%%%%%%%%%%%%%%%%%%%%%%%%%%%%%%%%%%%%%%%%%%%%%%%%%%%%%%%%%


%----------------------------------------------------------------------------------------
%	LETRAS LATINAS

% ENTRADA 1
\newglossaryentry{PoroPresAgua}{
name=\ensuremath{P_{w}},%SIMBOLO
description={Poro-presion del agua},%TERMINO
unit={$Pa$},%UNIDAD SI
def={Ecu \ref{eq:equ32}},%DEFINICION
type=symbolsLAT, 
sort={pw}}%ORDENAR

% ENTRADA 2
\newglossaryentry{PoroPresAire}{
name=\ensuremath{P_{a}},%SIMBOLO
description={Poro-presion del aire},%TERMINO
unit={$Pa$},%UNIDAD SI
def={Ecu \ref{eq:equ32}},%DEFINICION
type=symbolsLAT, 
sort={pa}}%ORDENAR

% ENTRADA 3
\newglossaryentry{permeabilidad}{
name=\ensuremath{k},% SIMBOLO
description={Permeabilidad absoluta},% TERMINO
unit={$m^2$},% UNIDAD SI
def={Ecu \ref{eq:equ311}},% DEFINICION
type=symbolsLAT, 
sort={k}}% ORDENAR


% ENTRADA 4
\newglossaryentry{ANNInput}{
name=\ensuremath{x_{i}},%SIMBOLO
description={Datos de entrada-ANN},%TERMINO
unit={},%UNIDAD SI
def={Ecu \ref{eq:equ32}},%DEFINICION
type=symbolsLAT, 
sort={xi}}%ORDENAR

% ENTRADA 5
\newglossaryentry{ANNPesos}{
name=\ensuremath{w_{ji}},%SIMBOLO
description={Pesos de conexión-ANN},%TERMINO
unit={},%UNIDAD SI
def={Ecu \ref{eq:equ32}},%DEFINICION
type=symbolsLAT, 
sort={wji}}%ORDENAR

%----------------------------------------------------------------------------------------


%----------------------------------------------------------------------------------------
%	LETRAS GRIEGAS

% ENTRADA 1
\newglossaryentry{viscosidad}{
name=\ensuremath{\mu},
description={Viscosidad dinámica},
unit={$Pa\cdot s$}, 
def={Ecu \ref{eq:equ32}},
type=symbolsGRI}

% ENTRADA 2
\newglossaryentry{porosidadVerdadera}{
name=\ensuremath{\phi},%SIMBOLO
description={Porosidad verdadera},%TERMINO
unit={$Adim$},% UNIDAD SI
def={Ecu \ref{eq:equ32}},
type=symbolsGRI}% ORDENAR

% ENTRADA 3
\newglossaryentry{CoeficienteBiot}{
name=\ensuremath{\alpha},%SIMBOLO
description={Coeficiente de Biot},%TERMINO
unit={$Adim$},% UNIDAD SI
def={Ecu \ref{eq:equ32}},
type=symbolsGRI}% ORDENAR

% ENTRADA 4
\newglossaryentry{poisson}{
name=\ensuremath{\nu},%SIMBOLO
description={Coeficiente de Poisson},%TERMINO
unit={$Adim$},% UNIDAD SI
def={Ecu \ref{eq:equ32}},
type=symbolsGRI}% ORDENAR

% ENTRADA 5
\newglossaryentry{divergente}{
name=\ensuremath{\nabla},%SIMBOLO
description={Operador divergente},%TERMINO
unit={},% UNIDAD SI
def={Ecu \ref{eq:equ32}},
type=symbolsGRI}% ORDENAR


%----------------------------------------------------------------------------------------


%----------------------------------------------------------------------------------------
%	SUBINDICES

% ENTRADA 1
\newglossaryentry{spcIndx}{
name={i,j,k},
description={Indices de espacio},
type=subIndx}

% ENTRADA 2
\newglossaryentry{spcIndx2}{
name={x,y,z},
description={Indices de espacio en el origen},
type=subIndx}

% ENTRADA 3
\newglossaryentry{spcIndx3}{
name={f},
description={Fluido},
type=subIndx}

% ENTRADA 4
\newglossaryentry{spcIndx4}{
name={s},
description={Particulas Solidas},
type=subIndx}

% ENTRADA 5
\newglossaryentry{spcIndx5}{
name={n,nw},
description={Fluido no-mojado},
type=subIndx}

% ENTRADA 6
\newglossaryentry{spcIndx6}{
name={w},
description={Fluido mojado},
type=subIndx}

% ENTRADA 7
\newglossaryentry{spcIndx7}{
name={p},
description={Presión de fluido},
type=subIndx}

% ENTRADA 8
\newglossaryentry{spcIndx8}{
name={t},
description={Derivada parcial respecto al tiempo},
type=subIndx}

% ENTRADA 9
\newglossaryentry{spcIndx9}{
name={$V$},
description={Volumen discretizado},
type=subIndx}



%----------------------------------------------------------------------------------------


%----------------------------------------------------------------------------------------
%	SUPERINDICES

% ENTRADA 1
\newglossaryentry{timIndx}{
name={t},
description={Indice de tiempo},
type=supIndx}

% ENTRADA 2
\newglossaryentry{timIndx2}{
name={0},
description={Tiempo inicial},
type=supIndx}


% ENTRADA 3
\newglossaryentry{timIndx3}{
name={u},
description={Cargas Nodales},
type=supIndx}

% ENTRADA 4
\newglossaryentry{timIndx4}{
name={p},
description={Termino fuente},
type=supIndx}

% ENTRADA 5
\newglossaryentry{timIndx5}{
name={$T$},
description={Transpuesta de la matriz},
type=supIndx}

%----------------------------------------------------------------------------------------


%----------------------------------------------------------------------------------------
%	ABREVIATURAS

% ENTRADA 1
\newacronym{ia}{IA}{Inteligencia Artificial}
\newacronym{ann}{ANN}{Redes Neuronales Artificiales}
\newacronym{edp}{EDP}{Ecuaciones Diferenciales Parciales}
\newacronym{fl}{FL}{Lógica Difusa}
\newacronym{mfe}{MFE}{Elementos Finitos Mixtos}

%----------------------------------------------------------------------------------------


\usepackage[titletoc,title]{appendix}

%-----------------------------------------------------------------------------------------
%				PROPIEDADES DE LA NUMERACIÓN DE ECUACIONES, FIGURAS Y TABLAS
\renewcommand{\theequation}{\thechapter-\arabic{equation}}
\renewcommand{\thefigure}{\textbf{\thechapter-\arabic{figure}}}
\renewcommand{\thetable}{\textbf{\thechapter-\arabic{table}}}
%\newcommand{\MYhref}[3][blue]{\hyperlink{#2}{\color{#1}{#3}}}
\allowdisplaybreaks

%-----------------------------------------------------------------------------------------
%					PROPIEDADES DE LOS ENCABEZADOS Y PIES E PAGINAS
\pagestyle{scrheadings}
\clearscrheadfoot

\newcommand{\mytitle}{Modelo geomecánico acoplado para flujo en medios porosos elastoplásticos}
\newcommand{\nomenclature}{Nomenclatura}

\rehead{\mytitle}
\lohead{\leftmark}
\ohead[\pagemark]{\pagemark}

\textheight22.5cm \topmargin0cm \textwidth16.5cm
\oddsidemargin0.5cm \evensidemargin-0.5cm%

\renewcommand*{\chaptermarkformat}{}
\renewcommand*{\sectionmarkformat}{}
\renewcommand*{\headfont}{\normalfont} 
\renewcommand*{\pnumfont}{\normalfont\bfseries}

\renewcommand*{\chapterheadstartvskip}{\vspace*{4cm}}
\renewcommand*{\chapterheadendvskip}{\vspace{1cm}}

%-----------------------------------------------------------------------------------------
%					         PSEUDOCÓDIGO Y ALGORITMOS
\usepackage{algorithmicx}
\usepackage{algorithm}
\usepackage{algcompatible}
\usepackage{algpseudocode}
\newcommand{\var}[1]{{\ttfamily#1}}% variable
\makeatletter
\renewcommand*{\ALG@name}{Algoritmo}
\makeatother
\renewcommand{\thealgorithm}{\arabic{chapter}.\arabic{algorithm}:}

%-----------------------------------------------------------------------------------------
%							  MARGENES DEL DOCUMENTO
\unitlength1mm %Define la unidad LE para Figuras
%\mathindent0cm %Define la distancia de las formulas al texto,  fleqn las descentra
\marginparwidth0cm
\parindent0cm %Define la distancia de la primera linea de un párrafo a la margen

%Para tablas,  redefine el backschlash en tablas donde se define la posición del texto en las casillas (con \centering \raggedright o \raggedleft)
\newcommand{\PreserveBackslash}[1]{\let\temp=\\#1\let\\=\temp}
\let\PBS=\PreserveBackslash

%Espacio entre lineas
\renewcommand{\baselinestretch}{1.15}
%\setlength{\textwidth}{500pt}
%%%%%%%%%%%%%%%%%%%%%%%%%%%%%%%%%%%%%%%%%%%%%%%%%%%%%%%%%%%%%%%%%%%%%%%%%%%%%%%%%%%%%%%%%%


%%%%%%%%%%%%%%%%%%%%%%%%%%%%%%%%%%%%%%%%%%%%%%%%%%%%%%%%%%%%%%%%%%%%%%%%%%%%%%%%%%%%%%%%%%
%-----------------------------------------------------------------------------------------
%							PROPIEDADES DEFINIDAS EN LA PLANTILLA
%Neuer Befehl f\"{u}r die Tabelle Eigenschaften der Aktivkohlen
\newcommand{\arr}[1]{\raisebox{1.5ex}[0cm][0cm]{#1}}

%Neue Kommandos
\usepackage{Varios/Befehle}

%Trennungsliste
\hyphenation {Reaktor-ab-me-ssun-gen Gas-zu-sa-mmen-set-zung
Raum-gesch-win-dig-keit Durch-fluss Stick-stoff-gemisch
Ad-sorp-tions-tem-pe-ra-tur Klein-schmidt
Kohlen-stoff-Mole-kular-siebe Py-rolysat-aus-beu-te
Trans-port-vor-gan-ge}
%-----------------------------------------------------------------------------------------
%%%%%%%%%%%%%%%%%%%%%%%%%%%%%%%%%%%%%%%%%%%%%%%%%%%%%%%%%%%%%%%%%%%%%%%%%%%%%%%%%%%%%%%%%%